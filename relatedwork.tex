\chapter{Background and Literature}
To set the scene, I first present a brief review of 3D printing's history, its working principles and some recent development of the technology. Next I shift the focus to academic research, where several trends of research have been capitalizing on the democratization of 3D printing, tackling problems related to optimizing the 3D printing process, making 3D printed objects more interactive, and innovating new types of fabrication tools or devices. As fabrication is a relatively new topic in Human-Computer Interaction (HCI), my hope is to draw together all this recent work to build up the backdrop of my thesis research. Last but not least, I focus on reviewing several papers that are specifically related to my research goals, which also prelude the following chapters where I describe my main results. \footnote{There is some other closely related work that will be reviewed inline with specific projects, as I feel the discussion of them is more appropriate within those contexts rather than in a standalone chapter.}.

\section{3D Printing: History, Principles and Recent Development}
The first 3D printing concept was published in 1981 by Hideo Kodama\footnote{\url{http://www.avplastics.co.uk/3d-printing-history}} from Nagoya Municipal Industrial Research Institute in Japan. He proposed a type of photopolymer rapid prototyping system that could build up layer after layer, each of which represented a slice of a 3D model. However, Kodama's idea did not seem to take off, as the first commercially available 3D printer only debuted 5 years later, which was a stereolithography printer by 3D Systems\footnote{\url{http://www.3dsystems.com/30-years-innovation}}.

\subsection{Major Categories of 3D Printers}
Today, the term `3D printing' loosely refers to a class of \textit{additive manufacturing} techniques\cite{gibson2010additive}. In general, the concept of additive manufacturing encompasses two key aspects: {\em (i)} the deposition of material through an \textit{extruder} (sometimes also called a print head); and {\em (ii)} the positioning of the extruder in 3D space. The combination of these two allow an object to be created following a bottom-up, layer by layer manner.
%FDM is a primary family amongst these techniques. Comparatively, FDM printer has fairly simple working principles and is relatively inexpensive to build. These machines would fabricate an object following a bottom-up, layer by layer approach. 

In an \textbf{Fused Deposition Modeling} (FDM) printer, an extruder deposits material (often called `filament') by melting and pushing it through a hot end. Each extrusion produces a thin line of material, which is then laid down on a print bed or some printed layers of an object. To position the deposition point, several motors drive the extruder and the print bed to move relatively to one another, allowing material to be laid down in pre-computed locations, forming (2D) geometric patterns. The printer will create a 2D layer at a given height, which represents a cross section of an object; then it moves vertically up to create the next layer above the current one. This process is repeated until a full 3D object is created from a stack of the 2D layers.

The \textbf{photopolymerization} approach, such as stereolithography (SLA) \cite{wiki:Stereolithography}, uses UV-curable photopolymer resin as material. UV light selectively cures a layer to create the cross-section of the object, and then the platform is lowered and the next layer is laid down; this process is repeated until all layers of an object is produced.

\textbf{Sintering}-based techniques, such as selective laser sintering (SLS) \cite{wiki:sls}, first lays down a layer of powder (e.g., metal) and then uses laser to selectively sinter part of the powder layer based on the geometric pattern of the cross-section . Similarly, next the platform is lowered and the next powder layer is laid down and sintered to continually build up an object.

\textbf{Laminated object manufacturing} (LOM) \cite{wiki:Laminated_objectmanufacturing} is an approach that successively glues together layers of material (paper, plastic or metal laminates), each of which, similarly represents a cross-section of a 3D object. Once printed, the object might be `trapped' inside the stacked layers (as volumetrically the object is only a subset of the collection of layers). Post processing might be needed to remove unwanted material and carve out the target object. Several recent projects have extended this approach, cutting and stacking fabric to create soft objects \cite{hudson2014printing, peng2015layered}.

\textbf{PolyJet}\footnote{\url{http://www.stratasys.com/3d-printers/technologies/polyjet-technology}}---a patented technology by Stratasys---is similar to inkjet printing except that liquid polymer is jetted, which is then cured, usually using UV light. The process is similar to a combination of FDM and material jetting (and curing).

%As simple as this process seems to be, it comes with certain limitation, such as larger layers might be not be supported by smaller layers beneath them, causing the object to `collapse'. In the next section, I will review existing solutions and research that address this problem.
%
%Besides FDM, there is a wide range of other additive manufacturing techniques; here I review several representative examples that have been widely adopted and used. Note that the general principle---layer-wise manufacturing of objects---is somewhat similar to FDM; what differs is often the types of material as well as the way the material is consolidated into an object's layers.

%At present, most consumer-grade 3D printers are FDM machines%\footnote{As FDM printers are so widely used by consumers, it has become the de facto `3D printers'. As such, I hereafter refer to the FDM technique as `3D printing' and FDM machines as `3D printers' }%. 

Note that the family of 3D printer is still growing and lots of innovation is taking place. For example, recent work by Harvard researchers can print metal structures in mid-air by annealing nanoparticles with laser \cite{skylar2016laser}. Perhaps the most innovative part of this work is it no longer constrains the printing process to layer by layer; rather, wire structure is created following its geometry, position and orientation in 3D space.

\subsection{Issues and Limitations}
3D printers' additive approach simplifies the making of objects with arbitrary shape by almost brute-forcedly treating them unexceptionally as a stack of layers. Such simplification also gives rise to several issues and limitations.

\textbf{Speed} is a universal issue amongst all types of 3D printers. Although the printing time varies from object to object, from printer to printer and from one setting to another, it is typical for a print job to take hours to finish. Fundamentally, the layerization of objects inevitably ignores their inherent geometric characteristics. Everything is sliced into a stack of layers, which at times is a suboptimal approach in terms of speed. Consider a long and thin vertical wire structure that contains very little material. A typical 3D printing process will treat the structure as a stack of its small cross sections, rather than optimally printing along its length (or height), where a strand of material can continuously make up a large portion of the wire's volume and in total less extruder movements are required to produce material for the object. Another reason for limited speed has to do with the deposition of material. For example, SLA needs to wait for each layer to cure in order to continue to the next one, making the curing time of material a bottleneck for speed. However, this is also an opportunity for breakthrough---recently Carbon\footnote{\url{http://carbon3d.com/}}, Prodways\footnote{\url{http://www.prodways.com/en/}}, SLASH\footnote{\url{https://www.kickstarter.com/projects/644653534/slash-the-next-level-of-affordable-professional-3d}} all claimed to have achieved much faster speed thanks to breakthrough in the chemistry of material that allows for shorter curing time.

\textbf{Overhang} Despite the universal simplicity of FDM, its layer-by-layer approach has a inherent limitation: each layer needs to `stand' on the previous ones. One can imagine that if a layer is much larger than the ones below it, part of this layer might have to stand on thin air, which is called an `overhang' situation. Overhang is not a universal issue---some 3D printing technology does not have this problem. For example, for SLA, SLS and LOM, the overhang structure is naturally supported by material that does not belong to the object but has not yet been removed during the printing time. FDM and PolyJet, however, does not have such `convenience' and will require a type of `support structure'---sacrificial structures that serve to support overhanging parts, and then removed from the final print. Essentially, support structures are put in place, and fabricated simultaneously with the target object to artificially create no-overhang situations. What is problematic about adding and printing these support structure is that it costs extra time and material to ensure stable support, exacerbating the speed problem. Further, as they become integral parts of the original object, it is generally difficult to remove them post hoc, unless the printer uses a separate, easily removable material specifically for support structure.

\textbf{Structural strength} affects the quality and usage of the 3D printed objects. Material is perhaps the primary factor for structural quality. Beyond traditional plastic-based filament, innovation in material, such as carbon nanotube\footnote{\url{http://www.3dxtech.com/}} and polycarbonate-based filament\footnote{\url{http://www.polymaker.com/shop/polymaker-pc-max/}} has proven to provide, amongst other improvements, stronger structural properties. However, material is not the only variable in deciding a printed object's strength. The layer by layer approach of FDM printers has caused a side effect of anisotropic strength--meaning the structural property of an object is dependent on the direction in which it is printed. Specifically, a printed object is generally much stronger within layers than between them, as adjacent layers are anisotropically adhered rather than isotropically integrated. 

Speed, overhang and structural strength are the three major issues/limitations for many of today's 3D printers. They also have aggregated a large body of research that seeks to solve or mitigate these problems, which I will review in the next section on 3D printing research.

\subsection{Printing with Multiple Material}
As many existing objects are made of more than one kind of material, it is a natural `push' for 3D printers to provide multi-material printing capability. Some types of printers are amenable to switching between different kinds of material. PolyJet, like inkjet printers, can select which color of material to use during the printing process. FDM, on the other hand, is not so flexible: to use the same extruder, the switching of material has to be done manually; alternatively, adding more extruders will inevitably increase the complexity of the mechanical design, which is not a scalable option. For LOM, although different layers can be arranged a priori, it only manages to change material from layer to layer, not within each layer. For SLS and SLA, even manually switching to a different kind of material is a cumbersome process, as it would require cleaning up the whole volume of current material and replenish it with new material.

\subsection{Cost and Customizability}
Even with simplified layer-wise working principles, the making of 3D printers was never a trivial task. Primarily, there have been two challenges: {\em (i)} precision---the thinner each layer could be the higher resolution the final object; {\em (ii)} material---the more material can be fed into a printer the more variety of objects users can make. The pursuit of high precision and special material has always been pushing the development of 3D printers.

Perhaps due to such a high technical bar (in both manufacturing and usage), 3D printers in their earlier years were never considered or envisioned as a piece of consumer-grade equipment. The expensiveness was another important factor. Indeed, according to 3D Printing Industry, not until 2007 did the market see the first 3D printer under \$10,000\footnote{\url{http://3dprintingindustry.com/3d-printing-basics-free-beginners-guide/history/}}. Even this price point was still way too expensive for consumer use. (In comparison, the first iPhone, which also debuted in 2007, was under \$600.)

Thus the dawn of making never took place in traditional 3D printing industry; in fact, many have attributed the democratization of 3D printing to some early do-it-yourself (DIY) projects that made the first steps of making affordable, (relatively) easy-to-build fabrication machines. For example, Fab@Home from Connell University is a kit available for general public to produce and assemble a multi-material 3D printer \cite{wiki:Fab@Home}. The RepRap project\footnote{\url{http://reprap.org/}} was engendered by Adrian Bowyer from the University of Bath in early 2004. The RepRap printer, foremost, is a fully functional FDM printer; further, the name stands for \textbf{Rep}licating \textbf{Rap}id-prototyper, meaning that this printer can be used to replicate new ones of its kind by producing constituent parts for assembly. This was a significant mission of this project, as it would provide a solution for making 3D printers widespreading to a large group of people.

Later in around 2009, Makerbot became one of the earliest consumer 3D printer manufacturers. In just a few years' time, numerous small or large companies emerged and joined the wave of making 3D printing affordable and usable to non-professional, non-expert users. To date, the cost of a consumer 3D printer has dropped to less than \$300. These low-cost 3D printers can be found in people's garages and basements, in educational and research institutes, and in hardware and facility stores. In 2013, Gartner projected a growth of 95.4\% in consumer and enterprise 3D printer shipments worldwide through 2017\footnote{\url{http://www.gartner.com/newsroom/id/3139118}}.

In academic research community, the availability of 3D printers not only makes it possible to bring 3D designs to reality, but also continuously inspires new ideas and problems, which I review in the following section.

\section{Current Research Trends in 3D Printing}
As mentioned above, 3D printing is a very specific type of fabrication technique with relatively simplified principles following a bottom-up, layer by layer approach. It currently stands at the intersection of several research areas, including mechanical engineering, material science, computational geometry, computer graphics, and HCI. My thesis research is primarily situated in the area of HCI, and partially related to computational geometry and computer graphics. The literature review in this section spans these three areas (primarily in HCI) with an attempt to weave prior work into a backdrop of my own research.

Within this scope, research on 3D printing by and large converges to the following three trends: {\em (i)} optimizing the 3D printing process, {\em (ii)} making 3D printed objects with custom properties, and {\em (iii)} innovating tools and devices for fabrication. While such categorization is by no means complete (nor are the trends mutually exclusive to one another), it serves well to organize a massively growing body of work that involves 3D printing.


\subsection{Optimizing the 3D Printing Process}
Although 3D printing can expressively fabricate objects of almost any shape, the process itself has some inherent limitations. 

\textbf{Speed} is perhaps the most noticeable problem. %As a rough estimate, for an object of average geometric complexity and a scale of $10\times10\times10$ $cm^3$, it would probably take several hours to print on an existing consumer-grade printer. (For commercial ones, it would probably take even longer as the precision is usually higher, meaning more layers to print for the same object).% 
%\xac{Removed the estimate. I am afraid giving specific examples would just get bogged down to more and more details: how complicated is the geometry, what printer is used, what's the setting, what's the filament, etc.} One direct consequence of low speed (or longer print time) is a slower cycle of design iteration: users have to wait a long period of time before being able to tangibly tell whether and how well the design works. This is against the general iterative design approach, in which it is imperative for people to quickly prototype, evaluate and continually develop their design. 3D printing has become a bottleneck that interrupts the flow of people's creation, reflection and improvement process.
To solve the speed problem, researchers have taken an approximation based approach--instead of following the default slow printing process, they try to modify and accelerate it at the cost of lowering the fidelity of the printed result. 

For example, faBrickation uses Lego bricks to replace parts and components that would otherwise need to be 3D printed \cite{mueller2014fabrickation}. In this way, assembling existing Lego bricks makes it faster to approximate the target object than printing it from scratch. The accompanying design tool converts an existing 3D model into a `low-fi' version consisting of Lego bricks and outputs instructions for assembly. As this approach sacrifices the fine resolution of the original model, it also allows users to specify parts or components that need to be precisely 3D printed: a combination of both yield a flexible way to accelerate the making of an object. Similarly, Platener introduces laser cutting---a much faster fabrication process---into 3D printing. The design tool looks for parts of an object that resemble planar structures, and uses laser cut `plates' to replace them, thus reducing the extra time if otherwise using 3D printing \cite{beyer2015platener}.

Besides approximation, it is also possible to print objects incrementally. For objects with large volume, researchers have come up with techniques that build these objects around a (large) existing component. For example, RevoMaker is a modified 3D printer wherein an existing block of material can be clamped on, rotated, and new parts can be printed on each side of it to finish a complete object \cite{gao2015revomaker}. CofiFab presents a coarse to fine approach, which first produces a coarse approximation of an object as the base (often using faster methods such as laser cutting), and subsequently fabricates parts with finer-grained details onto it \cite{songcofifab}.

A third approach of acceleration is to redesign the printing process. WirePrint, for example, converts a 3D model into a `low-fi' custom wire mesh structure amenable to be printed fast \cite{mueller2014wireprint}, especially when using printers with high Z-axis speed (e.g., the Delta printer\footnote{Instead of moving the print head orthogonally, Delta printers apply the delta robot principle, which uses three arms to position the print head in 3D space.}). Similar to how `low-poly' tessellation saves space for storing 3D models, `low-fi' printing saves time for making 3D objects. This approach was further developed in On-the-Fly Printing--a custom 3D printer that uses cooling to further accelerate the WirePrint approach and adds a rotation mechanism so that new `wires' can be added or removed at arbitrary points \cite{peng2016fly}. This enables incremental on-the-fly printing that happens as users are virtually building or editing a 3D model, thus accelerating the process with parallelization.

The requirement of \textbf{support structure} is another concerning problem of 3D printing. In the research community, there have been several approaches focusing on solving the first problem of support structure. Specifically, the goal is to reduce the cost of extra time and material without making these structure less stable (i.e., the overhanging layers can still stand through the 3D printing process). Schmidt and Umetani proposed a top-down procedural approach to generate support structures, which results in a significant reduction in wasted time and material compared to using manufacturer-provided method \cite{schmidt2014branching}. The tree structure in Schmidt and Umetani's work is further improved by Dumas et al., who propose a bridge structure to provide stronger and more stable support \cite{dumas2014bridging}. Bridging is a specific FDM technique wherein the printer extrudes material horizontally trying to connect two end points at the same level. Although this is also an overhanging situation, the bridge is usually able to stand because of the tension of the extruded material as well as the support at the two ends (similar to how a hammock can be tied to and hung between two structures and even support human weight). Dumas et al.'s work exploits this bridging capability to hierarchically build up steady scaffolding in a cost-efficient way.

\textbf{Structural quality} This unique problem of 3D printing has motivated several interesting projects, such as Umetani and Schmidt's cross-sectional analysis \cite{umetani2013cross} and Ulu et al.'s build orientation optimization \cite{ulu2015enhancing}, both of which attempt to find an optimal direction to print an object so as to maximize its strength (or minimize the effect of cross-layer anisotrophism).

All this work above focuses on understanding the limitations, optimizing the printing process and has solved a number of interesting problems related to speed, overhang and structural quality. Next I move on from the printing process and look at the design of objects that gives them custom properties once printed.

\subsection{Making 3D Printed Objects with Interactive Properties}
One compelling power of 3D printing is the customization of objects--the ability to transform them into something with unique properties customized by the users. 

One well-explored approach to achieve this is by embedding custom structure or components into a printed object. This approach can often result in more interactive 3D prints, as the embedding enables the objects to transmit various signals with input and output capabilities. For example, Printed Optics is a general technique that embeds optical elements into objects to serve as display or optoelectronic sensors that react to how people manipulate the objects \cite{willis2012printed, brockmeyer2013papillon}. InfraStruct expands this idea to the terahertz (or sub-millimeter) spectrum: through terahertz imaging information can be encoded into or read from the internal design of 3D printable objects \cite{willis2013infrastructs}. Accoustruments further pushes this approach with transmission of a near-ultrasound swept frequency signal through a custom-built embedded tube. Upon reception of the signal, it can characteristically describe users' interaction with a printed object. In particular, this approach seems more deployable to smart devices, as audio signals can be received by their microphones, and 3D printed enclosures can be easily attached to the devices \cite{laput2015acoustruments}. Similarly, Lamello 3D prints objects with structures that produce distinct acoustic signals when interacted with \cite{savage2015lamello}, such as a slider whose thumb brushes through a comb-like structure and produces distinguishable sound corresponding to its position.

It is also possible to embed and analyze other types of signal. Vazquez et al. manipulate air pressure inside a printed object with custom material and internal structure, which can also afford a similar class of interaction \cite{vazquez20153d}. Similarly, Schmitz et al. embed liquids into 3D Printed objects to sense tilting and motion \cite{schmitz2016liquido}. Vasilevitsky et al. embed structures within 3D printed mechanical components that enable the sensing of their mechanical states through electronic signal, such as a voltage divider in a gear and a variable capacitor in a hinger \cite{vasilevitsky2016steel}.

All this work above has pursued a close coupling of an object's intrinsic geometry and the added structures or components that give rise to new--mostly interactive---properties. In Graphics, researchers also have explored other physical properties of printed objects, such as making imbalanced objects stand \cite{prevost2013make}, or making static objects easily spinnable \cite{bacher2014spin}, both of which by optimizing their internal material distribution.

Other work also investigates a `looser' way of making 3D printed objects interactive via instrumenting extra components, often as a post processing step. For example, Sauron instruments mirrors inside a 3D printed object, such as a game controller, and uses a light source and a camera to detect user interaction via light redirection, such as sensing the button presses on different locations of the controller \cite{savage2013sauron}. Makers' Marks further facilitates the prototyping process using sculptable material (e.g., clay) to create physical models with add-on mechanical or electronic components. A vision-based approach then extracts the location information of these add-ons, which subsequently inform their final installation on 3D printed objects \cite{savage2015makers}. Jones et al. explored a similar approach, but instead building the fiducial marks as part the sculpted model, which then allow the scanning software to identify widget types and locations, and create structure and space for mounting them \cite{jones2016you}.

%\xac{how to scope this?}
%Another approach for making objects more interactive is to innovate on the printing material. There is more than one field of research that focuses on this approach, such as material science, bioengineering, and medical engineering. In HCI, a group of researchers from MIT Media Lab also experimented with new 3D printing material that can create interactive artifacts. Specifically, they use Bacillus subtilis---a type of bacterium that exhibits hygromophic behavior in response to changes in humidity. It is shown that this type of bacteria can be used to create bio-enabled actuators made of a composite of the cells, such as making the petals of a flowers unfold when watered, or making areas on clothing open when the wearer sweats \cite{yao2015biologic}. A follow-up project of this work further shows how to build an FDM styled printer (with a key component of a progressive cavity pump) that can utilize this type of material for fabrication \cite{wang2016xprint}.


\subsection{Innovating Tools and Devices for Fabrication}
The expressiveness (as well as limitations) of 3D printing has cultivated research on improving its process and diversifying the printed object's properties and behaviors. Further, it also inspired innovation on new fabrication tools and devices that are based on, or at times go beyond the traditional FDM approach.

One fairly customizable component is the extruder. For example, the `Teddy bear printer' is able to 3D print soft object with needle-felted yarn. While the architecture of the printer is largely based on a conventional FDM machine, the key innovation of a custom print head contributes to a vastly different material usage and property. Specifically, the custom print head is able to handle the extrusion of and the piercing and entangling with soft fibre, while leveraging the along-axis movement to formulate the geometry of print \cite{hudson2014printing}. The aforementioned xPrint project also innovates the extruder of a conventional FDM printer, allowing it to handle solution material, while also able to be plugged in with other modules, such as ventilation and mechanical stirs \cite{wang2016xprint}.

It is also possible to combine both subtractive and additive manufacturing into one single printer. Scotty is a realization of this concept: as an object is milled away, each layer is scanned; the digitalized model can then be encrypted, sent to another printer, and printed using the additive approach \cite{mueller2015scotty}.

Another direction is to extend the degree of freedom (DOF) of a conventional printer. One example of this approach is Peng et al.'s On-the-Fly printing---by adding two rotational DOF to a printer, it is now possible to print an object incrementally from various points and orientations, rather than always follow a fixed direction \cite{peng2016fly}.

Other researchers also attain to augment the `vision' of a 3D printer. Since there is no feedback loop in a conventional printer, the machine has no knowledge or awareness of the object that is being printed, thus no ability to prevent errors or improve upon its current state. To overcome this weakness, MixFab provides a mixed reality environment for users to model their design while juxtaposing it with real objects \cite{weichel2014mixfab}. To truly integrate `vision' into the printing process, MultiFab employs cameras and machine vision techniques to enable the printer to self-calibrate, scan object or partial prints, and perform self-corrections \cite{sitthi2015multifab}.


\section{3D Printing for Real Usage, with Real Objects, by Real People}
Amongst all this plethora of recent work, there is a strong focus on innovating the printing process, material and machine design, which enables more efficient printing, a larger variety of printable objects, and different fabrication techniques to empower and realize a user's creativity. While all this advancement supports an end-to-end pipeline of fabricating new things, it limitedly assumes that objects always have to be made from scratch. What about a world of things that already exist? Should they be discarded and replaced, or rather augmented and built upon? What usage and functional requirements should be considered when augmenting or in general dealing with  real world objects? How to specify these requirements in the design process? Further, lots of existing work also worships a complete automation of the fabrication process where sometimes the user's act of making is reduced to a few button presses. Does this limit people's innate creativity by keeping them `out of the loop'? In this section I review several key existing projects that investigate these questions - specifically, how should 3D printing involve real objects and real people.


\subsection{Combining 3D Printing and Real World Objects}
By and large, existing work has explored two ways of combining 3D printing and real world objects: {\em (i)} making objects that are composed of both 3D printed parts and existing things as components, and {\em (ii)} using 3D printing to augment real world objects.

\textbf{Composing target objects with real world objects.} Research that follows the first approach is perhaps motivated by the need to overcome the limited speed of 3D printing. If it is possible to substitute 3D printed parts with something that already exists, less of the object needs to be printed hence smaller amount of overall time. 

The aforementioned faBrickation project takes a quite radical approach substituting the majority of an object with lego bricks while only keeping precision-senstive components to be 3D printed \cite{mueller2014fabrickation}. For example, to make a headmounted phone case (for virtual reality applications), everything except the lens holder can be assembled using lego bricks. The design tool performs the conversion from the original geometry to lego-based composition and outputs instructions for assembly. It is obvious that lego bricks only provide fairly limited resolution; however, it manages to instantiate a rough object of a user's design, which is amenable to the iterative design process. Another project by Luo et al. goes beyond prototyping and attains to make functional object with lego bricks. In particular, they apply a force-based analysis to achieve a layout of lego bricks that makes the target object stable even at a fairly large scale \cite{luo2015legolization}.

Laser cut pieces can also be used to substitute 3D printed parts, which is demonstrated in the aforementioned Platner project  \cite{beyer2015platener}. It is worth noticing that this method is not limited to pure planar geometry: it is possible to laser cut curvy structure by, for example, using a special `S' cutting patterns, or making the piece bendable with post hoc heating.

To generalize all this work, the essential idea is anything can become part of something we wish to fabricate, at least geometrically. 3D Collage demonstrates this idea by allowing artist to compose a target geometry with a collection of other objects (which are often semantically related to the target) \cite{gal20073d}. RealFusion provides a 3D reconstructing environment wherein a user can simply bring in ready-made objects, digitalize and use them to model a target object \cite{piya2016}. To push this essential idea further towards fabrication, BottlePrint offers a design tool for using empty plastic bottles to approximate 3D objects, which allows making things at much larger scale beyond desktop printers. Virtually, they employ a tetrahedron-based tesselation sceme making target objects amenable to be made of by putting together and connecting bottles. Physically, a custom bottle connector is 3D printed, which is also numbered to ease the assembly process \cite{robertkovacs2016}.

\textbf{Using 3D printing to augment real world objects.} The previous group of work tends to treat everyday objects as building blocks deprived of their existing fuctions but serving to be components of something else. Another possibility to deal with these objects is to build upon their existing functionality and to agument or customize it with 3D printed parts. In fact, the need to augment everyday objects predated the dawn of personal fabrication. For example, Davidoff et al. propose `mechanical hijacking'---using motors that are designed to actuate existing controls or physical interface in specific ways \cite{davidoff2011mechanical}. However, these `hijacking' devices have to be manually designed. RetroFab, on the other hand, provides a design environment for end users to scan existing physical interfaces, design new add-ons to modify their controls, and fabricate and install them so as to automate or optimize the usage with these devices or appliances \cite{ramakers2016retrofab}. RetroFab's augmentation is generally heavyweight and aim for automating physical interfaces; in contrast, Project Reprise (detailed in Chapter 4) focuses on exploring a wide range of lightweight adaptations onto everyday things and hand tools to support custom usage or for users with special needs \cite{chen2016reprise}. What Reprise represents is a class of design tools that go beyond the specification of geometry alone to provide very application domain specific knowledge and features related to real world objects. As a result, the printed adaptation serves for specific real world functions in tandem with existing objects.

One fundamental issue of 3D printed augmentations is how to attach them to existing objects. A standard approach is to use fasteners or adhesives - using fasteners would require attachments to be designed with fasteners-specific structures, such as bolt holes; using adhesives is more universal and accessible but also takes non-trivial effort and expertise to attain a strong adhesion. With 3D printing, we can explore other attachment techniques by leveraging the customizable geometry and printing process. Project Encore (detailed in Chapter 3) was conducted to experiment with such 3D printing based attachment techniques. Around the same time, Teibrich et al. explored a similar idea of `patching physical object'. Their printer---augmented with milling bit and 5-dof movement---is able to mill out parts of an existing object so that new components can be readily printed on it \cite{teibrich2015patching}.

\subsection{Involving People into the Fabrication Process}
We have shown that researchers are trying to change the fact that 3D printing by default is a process detached to the a real world of objects. However, few have realized that even more detached is the involvement of real people. Although it looks like a dawn of fabrication is bringing `making' back to people, it is in fact  the machines that are doing the actual `making'. Indeed, oft-time what people need to do is no more than pressing a few buttons: open the 3D printing software, load a 3D model, hit `slice', and hit `print'. There is hardly any `making' involved in this process. It is true that having to design an object and use a 3D printer has gotten people further into the thinking of making ever than before, but the hands-on experience of making is still missing, so is the realization of the potential benefits of involving people in the process. As a result, people as makers are unfortunately very reliant on machine to execute a making job.

Fortunately, we have seen a few projects attempting to put people back into the loop of making. For example,  WrapIt enhance hand-made wrapped jewlery by providing 3D printed jigs based on digital design, which then guide designers to wrap wires into custom shapes to realize the design \cite{iarussi2015wrapit}. ProxyPrint further explores how 3D printed artifacts can serve not as the target objects themselves, but intermediate tools for users to handmake those objects \cite{torres2016proxyprint}. 

Researchers also developed novel hardware for promoting users' participation into the making process. D-coil provides an actuatable workbench as well as a hand-held wax extruder, together leading the user to extrude wax that formulates predefined geometry \cite{peng2015d}. Protopiper is hand-held device that extrudes plastic pipes as building blocks for users to prototype objects at room-scale \cite{agrawal2015protopiper}. Reform integrates many of the aforementioned elements--it combines both additive and subtractive manufacturing with built-in scanning capabilities. This allows a collaborative relationship between humans and machines: users initiate a design with intuitive hand-modeling of soft material, the printer then digitizes it and performs machine-based modification as the user iterates on the digital model \cite{weichel2015reform}.


\subsection{Specifying Real World Usage and Functional Requirement}
Functional specification is a very essential step in professional industrial design and manufacturing.

For non-expert users, however, it could be difficult to understand or incorporate such information into the modeling process. To overcome this difficulty, prior work has explored guided design of functional objects. SketchChair allows a user to sketch a chair by drawing its cross section on a 2D canvas then `extruding' it to a full 3D model \cite{saul2011sketchchair}. Further, their tool can test the chair using a virtual human model and physical simulation, so that users are aware of any potential problem of their design \cite{saul2011sketchchair}. However, there is little support for informing the users how they could correct or improve their design. To bridge this gap, Umetani et al. propose a design environment that, during the geometric editing process, also continuously visualizes the valid range of the design parameters \cite{umetani2012guided}. Specifically, \textit{feedback} is provided to the users once a constraint is violated, while \textit{suggestions} guide them to transform the problematic design to a valid one. Although the system provides useful and executable guidelines, the design tasks seem to be fairly limited, primarily focusing on arranging a set of rectangular planks. Martinez et al. enables more freedom of expressing a visual pattern by providing a user-defined template \cite{martinez2015structure}. By feeding this template into an optimization pipeline, the system can achieve aesthetically pleasing design while staying close to a strong enough structure. Although these patterns define a user's desired appearance of the object, they remain as microscopic features; there is little support for users to design the macro geometry of the object other than indirectly providing functional constraints to the optimization pipeline.

Outside of academic research, there is also a growing interest in industries to support end-user friendly functional optimization of their design. For example, Project DreamCatcher\footnote{\url{https://autodeskresearch.com/projects/dreamcatcher}} from Autodesk aims to generate a large number of design options based on users' high-level functional goals. It applies a generative design method to create design driven by specific "functional requirement, material type, manufacturability, performance criteria, and cost restrictions". In general, however, functional aspects of 3D design is yet to be supported in existing end-user oriented modeling tools.


\section{Summary of Literature Review}
In this chapter I lay out the historical context of 3D printing, briefly introduce its working principles and recent development into consumer-grade low-cost devices. This has fostered a new area of fabrication research spanning HCI, graphics, mechanical engineering and a few other domains. As the focus of my thesis is mostly in HCI (with some overlaps with geometry and graphics), I then sample a range of recent work in these areas as a way to elicit my particular interest in this thesis, which is to integrate 3D printing with real world objects and to involve real people into the making process. The next chapter unfolds a presentation of four projects that encompasses these two themes, starting with techniques for making 3D printed attachment directly onto, around and through real world objects.