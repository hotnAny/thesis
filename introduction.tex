\chapter{Introduction and Overview}

\section{The Disparity of Creativity}
As computers and the Internet become increasingly ubiquitous, people are now able to create or curate a variety of digital content. Today it only takes a few minutes to build and customize a personal website. Anyone can write a beautifully typeset blog without knowing anything about layout, font, or kerning. Writing and publishing apps has never been easier. You can even build your own virtual reality without making a step out of your dorm room\footnote{Minecraft. \url{https://minecraft.net/}}, or catch an army of Pokemons as you go\footnote{Pokemon Go. \url{http://www.pokemon.com/us/pokemon-video-games/pokemon-go/}}. The digital realm is such a wonderland where anyone, with a mouse, a keyboard or even just a touch screen, can become the greatest architect of their own virtual world.

In contrast to the prosperity of the cyberspace, people's ability to transform the real physical world has always been quite limited. In fact, since the second industrial revolution, people have been increasingly reliant on manufacturing and retail to procure what they want---the making and craftsmanship of individuals gradually became obselete and increasingly irrelevant. More recently, the convenience of online shopping and low-cost, fast shipping further disincentizes people from make things on their own.

Thus we are faced with a disparity of creativity: we are so freely and capable of building almost anything in the digital realm; we are so limited and much less capable of making things in the physical world.

\section{The Dawn of Making}
Fortunately, the advent of low-cost fabrication technology (most notably 3D printing) has given rise to a dawn of making. With 3D printing---a simple layer-by-layer additive manufacturing approach---people can simply import a 3D model, let the printer take care of the fabrication process and produce a physical artifact of their own design. Even without professional 3D design skills, people can go to online communities such as Thingiverse\footnote{Thingiverse. \url{http://www.thingiverse.com/}} to find and share 3D models with one another. These communities have curated a large set of things a person would want to make---from reconstructed model of Roman sculpture, to a fully functional plastic bottle shredder. Meanwhile, despite the current material limitation in desktop printers, we are moving towards a future where the availability of various materials further broadens the scope of making. People can now print things that are soft\footnote{NinjaFlex. \url{https://ninjatek.com/}}, sculptable\footnote{Sculptable material by Adam Beane Industries. \url{http://www.adambeaneindustries.com/}}, chemical-resistant\footnote{FilaOne GRAY by Avante Technology. \url{http://www.avante-technology.com/}}, structurally strong\footnote{PC-Max by Polymaker. \url{http://www.polymaker.com/shop/polymaker-pc-max/}}, magnetic\footnote{Polymagnets. \url{http://www.polymagnet.com/polymagnets/}}, metalic\footnote{Yibo3D. \url{http://www.yibo3d.com/}}, edible\footnote{Edible six-pack holder by We Believers \url{http://www.webelievers.com/}}, or self-actuating \cite{maccurdy2015printable}.

3D printing\footnote{As I focus on 3D printing as the main fabrication technique, hereon in this thesis I used the terms `3D printing' and `fabrication' interchangeably} has served as a platform that aggregates research and development from various domains to collectively enable people to transform their imagination and creativity into real objects that could impact their real lives. Indeed, such real world impact, whether it is small life hacking or a world-wide collaborative project \cite{schull2015enabling}, is what makes 3D printing stand out from other consumer software and electronics.

\section{The Oblivion of the Real World}
Given such fabrication capability, one natural approach to impact the real world is to use the printers to constantly produce new things--bespoke objects or people's own creative designs. While this has been an important aspect of 3D printing, the sole focus of making new things is oblivious of a whole world of existing objects that we are already interacting with at a daily basis. Rather than always producing things from scratch, perhaps another important mission of 3D printing is to closely couple with and build upon real world objects. In this way I believe we can maximize our `degree of freedom' in making things: people can choose to fabricate something entirely new, or to base their design on--and times in combination with-- existing things.

Besides existing objects, another important aspect that is often ignored in fabrication is people. It is true that 3D printing is bringing people back to making things. The kind of making it affords, however, is quite different from traditional practices. Traditional fabrication--everything from jewelry making to wood working--is extremely hands-on. Even later using modern tools and machinery, this making process still requires makers to closely interact with, control or manipulate physical material or artifacts. In contrast, 3D printing, especially the actual printing process, involves very little of such hands-on exercise. Minimally, a person only needs to press a few buttons to fabricate an object: download a 3D model, convert it to machine-readable code, and send it to the printer. In such a way of making, the real maker is not the people but the machine: there is currently no way people can contribute to the printing process once the `Print' button is pressed.

\section{Research Goals}
The goal of my thesis research, in brief, is to \textit{Make Fabrication Real}-- {\em (i)} my primary goal is to enable a making process (executed by 3D printers) that is closely integrated with real world objects and usages and; {\em (ii)} my secondary goal is to explore ways whereby the making process can benefit from real people's involvement.

To achieve the first goal, I build design tools that provide support for people to harness 3D printing not just to make things from scratch, but to extend, adapt or combine them with existing objects for individualized needs or custom use cases. To achieve the second goal, I explore fabrication techniques that invite and enable people to `get their hands dirty' and more actively participate in the making process; meanwhile, 3D printing `fades' into the background, acting as a supporting role to facilitate this process.

These research goals are manifested in four projects, as described in the remainder of this thesis.

\begin{itemize}
  \item \textbf{Chapter 2} sets the scene with a brief review on the background of 3D printing--its core technical components and principles, as well as some recent developments in both research and development.
  \item \textbf{Chapter 3} addresses the attachment problem, which is a fundamental requirement for using 3D printing to augment real world objects. Specifically, I present Encore---a design tool with a suite of techniques that allow a user, with a unmodified consumer grade 3D printer, to directly print attachments over, around or through existing objects.
  \item \textbf{Chapter 4} builds on the attachment techniques, and focuses on enabling the design tasks of making such attachments that work with existing objects. Specifically, I describe Reprise---a design tool that provides simple interaction techniques and computational geometry for specifying, generating, customizing and fitting adaptations onto existing objects.

  \item \textbf{Chapter 5}: Both Encore and Reprise focus on extending or adapting real world objects, which allows people to transform objects in their everyday lives. However, such transformation only adds small `delta' as it never goes beyond what existing objects were originally designed for.
  \\
  As a next step, I propose Mashup---a design tool that takes a mixed-initiative approach to guide users to compose and fabricate self-contained objects based on the functional requirements they will face once installed and deployed in the real world. Specifically, users start with sketching a design from their intuition; the system in the background optimizes their initial design based on the functional requirements; then users can further tweak the design while the system provides on-demand feedback and suggestion for keeping it functional.

  \item \textbf{Chapter 6}: Now that the range of making has broadened from incremental augmentation to standalone functional objects, a one-shot 3D printing job might no longer be the preferred making solution.

  My next proposed tool will seek to support people's hand making, assembling, installing and testing of functional objects in the real world. Specifically, I will develop Divje--a virtual tool using augmented reality to guide, assist or provide custom support for people during their manual fabrication process. Divje provides a unified making platform that integrates 3D design and modeling into hand making with material and artifacts. With Divje, people are immersed into both the digital and the real world of making: they can use virtual tools to sketch a design, which guides them to use physical material to make the corresponding prototype; as they build up the physical work they can also in parallel refine the model digitally. Instead of switching back and forth between the digital and the physical, Divje blends them together and presents the best of both worlds to enable people to bring their design to life with their hands.

  \item \textbf{Chapter 7} concludes this thesis by reflecting on my exploration of these projects, discussing issues and questions that arise, and suggesting new opportunities for future work.
\end{itemize}